\renewcommand{\.}{. \;}
\theoremstyle{definition}
\newtheorem*{definition}{Definition}
\newtheorem*{example}{Example}

\theoremstyle{plain}
\newtheorem{theorem}{Theorem}

\theoremstyle{remark}
\newtheorem*{remark}{Remark}

\newcommand{\ket}[1]{\lvert #1 \rangle}
\newcommand{\vc}[2]{\begin{pmatrix}#1\\#2\end{pmatrix}}
\newcommand{\vcc}[4]{\begin{pmatrix}#1\\#2\\#3\\#4\end{pmatrix}}
\newcommand{\op}[1]{\operatorname{#1}\;}
\newcommand{\oper}[2]{\begin{pmatrix}#1\\#2\end{pmatrix}}

\newcommand{\oort}{\frac{1}{\sqrt{2}}}



\tikzstyle{grammar}=[shorten >=1pt,->,draw=black!50]
\tikzstyle{nont}=[rectangle,inner sep=5pt, draw, fill=white, minimum width=20pt]
\tikzstyle{t}=[circle,inner sep=3pt, draw, fill=white, minimum width=25pt]
\tikzstyle{g}=[inner sep=3pt, fill=white]
\tikzstyle{lhs}=[inner sep=5pt, fill=white]
\tikzstyle{w}=[circle, inner sep=2pt, below=8pt, draw, fill=white]
\tikzstyle{edge}=[->, very thick, -latex]
\tikzstyle{wedge}=[-, thick]
\tikzstyle{biedge}=[-, very thick]
\tikzstyle{pipe}=[-, very thick]
\tikzstyle{morph}=[-, very thick, dashed, -latex]
